\documentclass{article}

\usepackage[utf8]{inputenc}
\usepackage{textcomp}

\begin{document}

\title{Grundlagen und Gegenmaßenahmen der Geldwäschebekämpfung}
\author{Gruppe}
\maketitle

\tableofcontents

\newpage

\part[Einführung]{Einführung}
    \section[Defintionen]{Definitionen}
        \subsection[Was ist Geldwäsche?]{Was ist Geldwäsche?}
            \texttt{Beschreibung, Zweck, Ablauf}
            
            \paragraph[Beschreibung]{Beschreibung}
            Als “Geldwäsche” wird das Einschleusen von Vermögenswerten, die aus einem Verbrechen oder aus bestimmten Straftaten herrühren (Vortaten), in den legalen Finanz- und Wirtschaftskreislauf, unter Verschleierung ihrer Herkunft, bezeichnet. Zu den Vortaten zählen z.B. Geldfälschung, Erpressung, Drogendelikte sowie Betrug, Korruption, organisierte Kriminalität, Terrorismus etc. Vortaten zur Geldwäsche werden durch lokale Gesetze definiert.
            
            \paragraph[Zweck]{Zweck}
            Rückführung von kriminiell erwirtschaftetem Vermögen in den Wirtschaftskreislauf.

            \paragraph[Ablauf]{Ablauf}

                \begin{enumerate}

                    \item Platzierung

                        Die Einschleusung illegal erworbener Mittel über Banken oder andere Institutionen.
                    \item Verschleierung

                        Die “Trennung” der illegalen Mittel von ihrer Herkunft durch diverse teils komplexe Finanztransaktionen. Diese sollen dazu dienen, die Nachvollziehbarkeit zu erschweren und für Anonymität zu sorgen.

                    \item Integration

                        Die Rückführung der “gewaschenen” Mittel in die legale Wirtschaft, so dass sie wie legale Mittel erscheinen.

                        \begin{enumerate}
                            \item Rechtfertigung
                            \item Investition
                        \end{enumerate}

                \end{enumerate}


        \subsection[Beispiel]{Beispiel}
            \texttt{Ablauf an Beispiel}

\newpage

\part[Bekämpfung]{Geldwäschebekämpfung}

    \section[Organe]{Organe}

        \texttt{Welche Institutionen sind an der Geldwäschebekämpfung beteiligt.}

        Bundesministerium der Finanzen

    \section[Richtlinien und Vorgehensweisen]{Richtlinien und Vorgehensweisen}

        \paragraph[Richtlinien]{Richtlinien}
            Welche Vorraussetzung bedingt es bestimmte Vorgehen zu initialisieren.

        \paragraph[Vorgehensweisen]{Vorgehensweisen}
            Nach Erfüllung bestimmter Vorraussetzungen machen ich bestimmte Dinge.

    \section[Probleme]{Problemen}

        \texttt{Lücken und Verbesserungen?}

\end{document}