\documentclass{article}

\usepackage[utf8]{inputenc}
\usepackage{textcomp}
\usepackage{graphicx}
	\DeclareGraphicsExtensions{.png}
\begin{document}

\title{Grundlagen und Gegenmaßenahmen der Geldwäschebekämpfung}
\author{Gruppe}
\maketitle

\tableofcontents

\newpage

\part[Einführung]{Einführung}
    \section[Defintionen]{Definitionen}
        \subsection[Was ist Geldwäsche?]{Was ist Geldwäsche?}
            \texttt{Beschreibung, Zweck, Ablauf}
            
            \paragraph[Beschreibung]{Beschreibung}
             	\texttt
                        {Als "Geldwäsche" wird das Einschleusen von Vermögenswerten, die aus einem Verbrechen oder aus bestimmten Straftaten herrühren (Vortaten), in den legalen Finanz- und Wirtschaftskreislauf, unter Verschleierung ihrer Herkunft, bezeichnet. Zu den Vortaten zählen z.B. Geldfälschung, Erpressung, Drogendelikte sowie Betrug, Korruption, organisierte Kriminalität, Terrorismus etc. Vortaten zur Geldwäsche werden durch lokale Gesetze definiert.}   

            		Als "Geldwäsche" bzw. "Geldwäscherei (in Österreich \& Schweiz)" werden sämtliche Handlungen bezeichnet, die das Ziel besitzen, die Herkunft illegal erlangten Vermögens zu verschleiern, um diese später unter anscheinender Legalität im Wirtschaftsverkehr benutzen zu können.

	               	Das wesensbildende Merkmal des Begriffs "Geldwäsche" besteht folglich darin, dass der Ursprung der Vermögensgegenstände direkt oder indirekt auf eine kriminelle d.h. illegale und strafbare Handlung zurückzuführen ist.
            
            \paragraph[Zweck]{Zweck}
            	\texttt
                        {Rückführung von kriminiell erwirtschaftetem Vermögen in den Wirtschaftskreislauf.}

	               	Der Geldwäsche liegt das Ziel zugrunde, die unrechtmäßige Herkunft der Vermögensgegenstände zu verschleiern, um ihnen nach Durchlaufen verschiedener Transaktionen den Anschein der Rechtmäßigkeit zu geben.
            		Neben der bewussten Vertuschung der illegalen Herkunft geht es bei der Geldwäsche außerdem darum, die Beschlagnahme der unrechtmäßig erlangten Vermögensgegenstände durch die staatlichen Strafverfolgungsbehörden zu verhindern.

            \paragraph[Ablauf]{Ablauf}
                \begin{enumerate}
                    \item Platzierung
                        \texttt
                            {Die Einschleusung illegal erworbener Mittel über Banken oder andere Institutionen.}

			         Unter der Platzierung (Placement) versteht man die Verschleierung der Herkunftsspuren des illegal erworbenen Geldes. Ferner soll die Identifikation sowie die Einziehung aus der kriminellen Tätigkeit durch den Staat verhindert werden. Die häufigsten Formen der Platzierung sind:
    		         \begin{enumerate}
				        \item
				        	die zu verschleiernde Geldmenge wird in kleineren Teilbeträgen in der Regel ohne zeitlichen Zusammenhang auf Banken, Geldwechselstuben, Spielkasinos, Münzhändler oder Wertpapiermakler verteilt.
        				\item
        					"Investition" des zu verschleiernden Geldbetrags in andere Vermögenswerte. Hierzu werden häufig Grundstücke, Luxusautos, Yachten, Schmuck etc. verwendet.
        				\item
        					Darlehensgewährungen an von den Beteiligten der Vortat nahe stehende Personen, Personenvereinigungen oder Institutionen.
        				\item
        					Beteiligungen an (stillen) Gesellschaften, die von den Beteiligten der Vortat beherrscht bzw. eingerichtet wurden.
        				\item
        					Versteuerung als legal erwirtschaftete Umsätze in Betrieben, die den weiten Teil ihrer Umsätze mit Bargeld tätigen. Anfällig hierfür sind insbesondere der Gastronomiebereich, Taxibetriebe und auch In- und Exportfirmen.
					
                    \end{enumerate}

                    \item Verteilung
                        \texttt
                            {Die "Trennung" der illegalen Mittel von ihrer Herkunft durch diverse teils komplexe Finanztransaktionen. Diese sollen dazu dienen, die Nachvollziehbarkeit zu erschweren und für Anonymität zu sorgen.}
			
			         In der Stufe der Verteilung (Layering) werden die illegal erworbenen Geldbeträge in andere Staaten transferiert. Hierbei kommen folgende Begebenheiten im Ausland den Beteiligten an der Vortat zu Gute:
			
        		  	\begin{enumerate}
        				\item
        					fehlende Buchführungspflicht,
        				\item
        					fehlende Bankaufsicht,
        				\item
        					mangelhafte Steuerkontrolle,
        				\item
        					mangelhafte Strafvorschriften sowie
        				\item
        					Vorhandensein von Domizil- und Finanzgesellschaften, welche kaum oder keinen Kontrollmechanismen unterliegen.
        			\end{enumerate}
        			
			Beim Transfer des Geldes werden in der Regel unverdächtige Dritte (Anwälte, Treuhänder oder Scheinfirmen) zwischengeschaltet, die durch die länderspezifischen Finanzgeschäfte und eine Vielzahl von Transaktionen für Verwirrung sorgen und eine erschwerte Verfolgbarkeit der illegalen Erlöse nach sich ziehen. Dabei werden Berufsgeheimnis und Verschwiegenheitspflichten geschickt für die eigenen Zwecke eingesetzt.
	
                    \item Integration
			             \texttt {Die Rückführung der "gewaschenen" Mittel in die legale Wirtschaft, so dass sie wie legale Mittel erscheinen.}
			
			Durch die Maßnahmen der Beteiligten der Vortat im Rahmen der Integration (Integration) fließt das in der Herkunft verschleierte Geld in den legalen Wirtschaftskreislauf zurück und erhält einen legalen Anschein, als sei es auf rechtmäßigem und nachvollziehbarem Wege durch die Ausübung einer geschäftlichen Tätigkeit erworben worden. Auch hier bedienen sich die Täter der für die Geldwäsche besonders geeigneten Bargeldbetriebe, wie z.B. Taxibetriebe, Gastronomiebetriebe sowie Im- und Exportfirmen.

			Bei der Rückführung des Kapitals in den legalen Wirtschaftskreislauf wird in der Regel auch eine Versteuerung hingenommen. Da der Geldwäscher im Allgemeinen eine gewinnbringende 	Anlage anstrebt, wird auch eine Steuerhinterziehung in Kauf genommen oder als Nebenziel angesehen.

			Banken und Kreditinstitute spielen für alle drei Stufen der Geldwäsche eine bedeutende Rolle. Garantierte Diskretion, Bank- und Berufsgeheimnis bieten unentbehrliche Vorteile für den Geldwäscher. Für eine Platzierung der Bargeldbeträge reicht schon eine Zwischenlagerung in Depots oder Bankschließfächern aus. Bankeigene Finanzdienstleistungen werden zur Verteilung benutzt und Darlehenstransaktionen, Akkreditive sowie die finanzielle Durchführung von Investitionsprojekten und die einfache Ausstellung von Schecks bietet den Geldwäschern Hilfestellung bei der Integration. Das Ganze erfolgt unter Ausnutzung der Stärken und Schwächen der Organisationsstrukturen und Überwachungsschwächen einzelner, für Geldwäsche besonders anfällige Länder.	

			Wichtigstes Hilfsmittel zur Bekämpfung der Geldwäsche sind die in vielen Ländern geltenden Schwellenwerte für Bargeldtransaktionen. Dies soll die Verteilung großer Mengen an Bargeld erschweren.

                        \begin{enumerate}
                            \item Rechtfertigung
                            \item Investition
                        \end{enumerate}

                \end{enumerate}


        \subsection[Beispiel]{Beispiel}
            \texttt{Ablauf an Beispiel}

\newpage

\part[Bekämpfung]{Geldwäschebekämpfung}

    \section[Organe]{Organe}


        \subsection[Organe BRD]{Organe innerhalb der BRD}

            \begin{enumerate}

                \item Das Bundeskriminalamt – Zentralstelle für Verdachtsmeldungen

                    Für: Hauptanlaufstelle für meldepflichtige Anzeigen.

                \item Bundesministerium der Finanzen

                    Für: ???

                    Kann selbst oder in Zusammenarbeit mit BdI Umstände bzw Art und Weise von Meldungen bestimmen (Verordnungen?!).

                \item Bundesministerium des Innern

                    Für: ???

                    Kann selbst oder in Zusammenarbeit mit BdF Umstände bzw Art und Weise von Meldungen bestimmen (Verordnungen?!).                

                \item Finanzagentur GmbH das Bundesministerium der Finanzen

                     Für: Kreditanstalt für Wiederaufbau und die Bundesrepublik Deutschland

                \item Bundesanstalt für Finanzdienstleistungsaufsicht
                   
                    Für: 
                    \begin{enumerate}
                        \item
                            
                            Noch nicht abgedeckte Kreditinstitute mit Ausnahme der Deutschen Bundesbank

                        \item

                            Finanzdienstleistungsinstitute und Institute im Sinne des § 1 Absatz
                            2a des Zahlungsdiensteaufsichtsgesetzes.

                        \item

                            Im Inland gelegene Zweigstellen und Zweigniederlassungen von
                            Kreditinstituten, Finanzdienstleistungsinstituten und
                            Zahlungsinstituten mit Sitz im Ausland.

                        \item

                            Investmentaktiengesellschaften im Sinne des § 2 Absatz 5 des
                            Investmentgesetzes´.

                        \item

                            Kapitalanlagegesellschaften im Sinne des § 2 Absatz 6 des
                            Investmentgesetzes.

                        \item

                            Im Inland gelegene Zweigniederlassungen von EU-
                            Verwaltungsgesellschaften im Sinne des § 2 Absatz 6a des
                            Investmentgesetzes.

                        \item

                            Die Agenten und E-Geld-Agenten im Sinne des § 2 Absatz 1 Nummer 2b.

                        \item

                            Unternehmen und Personen im Sinne des § 2 Absatz 1 Nummer 2c.

                    \end{enumerate}

                \item Jeweils zuständige Aufsichtsbehörde für das Versicherungswesen

                    Für: Versicherungsunternehmen und die im Inland gelegenen Niederlassungen solcher Unternehmen.

                \item Örtliche Rechtsanwaltskammer (§§ 60, 61 der Bundesrechtsanwaltsordnung)
                    Für: Rechtsanwälte und Kammerrechtsbeistände

                \item  Patentsanwaltskammer (§ 53 der Patentanwaltsordnung)
                    Für: Patentanwälte

                \item Präsident des Landgerichts, in dessen Bezirk der Notar seinen Sitz hat (§ 92 Nr. 1 der Bundesnotarordnung)

                    Für: Notare der jeweilige 

                \item Wirtschaftsprüferkammer (§ 57 Abs. 2 Nr. 17 der Wirtschaftsprüferordnung)
                Für: Wirtschaftsprüfer und vereidigte Buchprüfer

                \item Örtlich zuständige Steuerberaterkammer (§ 76 des Steuerberatungsgesetzes)

                    Für: Steuerberater und Steuerbevollmächtigte

                \item Im Übrigen, nach Bundes- oder Landesrecht zuständige Stelle ???

            \end{enumerate}

        \subsection[Organe EU]{Organe der EU}

            \begin{enumerate}

                \item Europäischen Bankenaufsichtsbehörde 

                \item Europäischen Aufsichtsbehörde für das Versicherungswesen und die betriebliche Altersversorgung 

                \item Europäischen Wertpapier- und Marktaufsichtsbehörde 

            \end{enumerate}        


    \section[Richtlinien und Vorgehensweisen]{Richtlinien und Vorgehensweisen}

        \paragraph[Richtlinien]{Richtlinien}
            Welche Vorraussetzung bedingt es bestimmte Vorgehen zu initialisieren.

        \paragraph[Vorgehensweisen]{Vorgehensweisen}
            Nach Erfüllung bestimmter Vorraussetzungen machen ich bestimmte Dinge.

    \section[Probleme]{Problemen}

        \texttt{Lücken und Verbesserungen?}

\end{document}
